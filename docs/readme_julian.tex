\documentclass[11pt,a4paper,notitlepage]{article}

\usepackage[utf8]{inputenc}
\usepackage[english]{babel}
\usepackage{titling}

\usepackage{authblk}
\usepackage{graphicx}
\usepackage{titlepic}
\usepackage{amsmath}
\usepackage{amsfonts}
\usepackage{amssymb}
\usepackage{mathtools}
\usepackage{graphicx}
\usepackage{tabu}
\usepackage{mathpazo}
\usepackage{stackengine}

%%% Spaces before and after sections
\usepackage{titlesec}

\titlespacing\section{0pt}{0pt plus 4pt minus 2pt}{0pt plus 2pt minus 2pt}
\titlespacing\subsection{0pt}{0pt plus 4pt minus 2pt}{0pt plus 2pt minus 2pt}
\titlespacing\subsubsection{0pt}{0pt plus 4pt minus 2pt}{0pt plus 2pt minus 2pt}

%%% PAGE DIMENSIONS
\usepackage{geometry}
\geometry{a4paper}
\geometry{margin=3cm}
\setlength{\parskip}{\baselineskip}
\setlength{\parindent}{0pt}


%%% HEADER

\usepackage{fancyhdr} % This should be set AFTER setting up the page geometry
\pagestyle{fancy} % options: empty , plain , fancy
\lhead{}\chead{}\rhead{}
\lfoot{}\cfoot{\thepage}\rfoot{}

\fancyhead[L]{\slshape {
		\includegraphics[scale = .2]{unisg_logo.png}
}}
\fancyhead[R]{\slshape {by DataLab St.Gallen}}

\title{Analyticathlon. St.Gallen's first Data Science Hackathon}
\date{\vspace{-5ex}}
\author{\vspace{-5ex}}

% --------------------------------------------------------

\begin{document}
\maketitle
\thispagestyle{fancy}

\section*{Why The Analyticathlon?}
 Data science currently is driving the biggest changes in the world, however, it still remains a black box to many. It is indeed difficult to understand both the applications and implications of data science without prior experience in the field.
 
 For example, it is hard to see how clean data can add value to a company if one has not tidied datasets and spent hours contemplating the way different statistical models can be applied to tease meaning out of the data.
 
 Te clear the fog surrounding the profession that gained recognition over the past decade, we propose hackathon focussed on solving data-related problems.
 
 St. Gallen's very first Analythicathlon is planned to take place in the late fall of 2018, over the span of three days. Students from all over Switzerland will be given the opportunity to analyze different datasets, with the aim of turning numbers into business insights.
 
\section*{Painting The Picture}
There will be around 80-100 students, professors and professionals in the field gathering in the newest coworking space in St. Gallen, theCo to exchange and learn about data science. The overarching task during the event will be to gain new insights and answer business questions from raw datasets. 

The students will receive help by experts from big technology companies, such as IBM, Google annd Microsoft, who will in turn get the chance to meet Switzerland's brightest analytical minds. 

Professors and students will be able to exchange their experiences in studying data science. Throughout the event, different workshops concerning analytical methods and tools will be offered by sponsors to help students in their work with the given data.

The event will be rounded up with a presentation of the findings as well as an explanation of new insights, methods and methodologies gained during the course of the hack.

\section*{Who’s Coming?}
\subsection*{Participants}
The event is open to students from all over Switzerland, while the preferred choice are those enrolled in a data science degree. Thus we will invite students from EPFL, ETH, UZH, ZHAW and other notable data science programs.

\subsection*{Guests}
Among the students and industry professionals attending the event, professors of the different data science degrees are welcome. Educators shall have the opportunity to exchange and connect with students, and fellows of their metier.

\subsection*{Sponsors}
The event will be sponsored by companies which take one of three roles. 

\textbf{Event Supporters.} As in every event there are normal sponsors, which provide financial means and utilities for the event. 

\textbf{Analytics Coaches.} Companies will receive the opportunity to send coaches to the event which can help the students in tackling the tasks, giving workshops about certain technologies and foremost promote their company to the brightest minds in Switzerland.

\textbf{Analytics Addressees.} The third type of sponsors will be companies which provide an object of research i.e. data and questions which they hope to answer with the data. The data will be evaluated for their adequacy and opportunity for analysis by experts before the event. It is critical, that the data provided shall have a large number of observations and variables. The companies we are looking for in this third sponsoring category are enterprises which have promising information but cannot yet put is to use effectively.

\section*{Value Added}
\subsubsection*{Building The Community}
Foremost, the event shall give students the opportunity to deep dive into applied data science and work on a project with real data provided by one of our sponsors.

The students aim to either answer questions supplied by the company with the data set or generate their own insights about the dataset.

In addition, the gathering of students and professors of Data Science in Switzerland will allow the exchange and community building between programs. 

\subsubsection*{Promoting The Field}
The Analyticathlon will promote multiple stakeholders. The event will brand the University of St.Gallen (HSG) as a “more than business environment” for the first time. In addition, the university will be viewed as a location where business leadership is combined with, and driven by data analytics. In this sense, the digital element is added to St.Gallen's integrative approach.

Furthermore, the Analyticathlon will support the advancement and promotion of the newly established Data Science Fundamentals Certificate, the secret ingredient in the education of decision-makers.

Building on the successful first implementation of the Data Science Fundamentals course students from HSG will be able to use their acquired skills in a real case and show how the additional study of data science adds value to the educational profile of business students.

Since the datasets will be provided by Swiss companies the participants will immediately create value for the Swiss economy and partnering firms.

Finally, The presence of big technology companies at the university will for the first time promote the idea of having HSG students as potential employees in the technology sector.

\end{document}